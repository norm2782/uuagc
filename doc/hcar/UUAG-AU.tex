\begin{hcarentry}[updated]{UUAG}
\report{Arie Middelkoop}%05/09
\status{stable, maintained}
\participants{ST Group of Utrecht University}
\makeheader

UUAG is the \emph{Utrecht University Attribute Grammar} system. It is a preprocessor for Haskell which makes it easy to write \emph{catamorphisms} (that is, functions that do to any datatype what \emph{foldr} does to lists). You can define tree walks using the intuitive concepts of \emph{inherited} and \emph{synthesized attributes}, while keeping the full expressive power of Haskell. The generated tree walks are \emph{efficient} in both space and time.

Idiomatic tree computations are neatly expressed in terms of copy, default, and collection rules. Computed results can masquerade as subtrees and be analyzed accordingly. The order in which to visit the tree is derived automatically from the attribute computations; the tree walk is a single traversal from the perspective of the programmer.

The system is in use by a variety of large and small projects, such as the Utrecht Haskell Compiler UHC, the editor Proxima for structured documents, the Helium compiler~\cref{helium}, the Generic Haskell compiler, and UUAG itself. The current version is 0.9.12 (October 2010), is extensively tested, and is available on Hackage.

We recently added support for building AG files through Cabal. A small Cabal plugin is installed upon installation of UUAG, which provides a userhook that deals with AG files and their dependencies.

\FurtherReading
\begin{compactitem}
\item
  \url{http://www.cs.uu.nl/wiki/bin/view/HUT/AttributeGrammarSystem}
\item
  \url{http://hackage.haskell.org/package/uuagc-0.9.12}
\end{compactitem}
\end{hcarentry}
